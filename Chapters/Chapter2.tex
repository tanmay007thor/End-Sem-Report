\begin{enumerate}
    \item[\textbf{[1]}] The proposed architecture, Efficient Sleep Sequence Network (ESSN), has overcome the limitations of existing automatic sleep stage algorithms. This model addresses two main challenges. First, the model is quite complex, and often low-end systems are unable to process it; therefore, this model is designed to work on lightweight systems. The second challenge is the misclassification of the N1 stage, where models often confuse wake and REM stages. To address this, it introduces the N1 structure loss function. The ESSN model has achieved impressive metrics: 88.0\% accuracy, 81.2\% macro F1, and 0.831 Cohen’s kappa. These results were obtained on the SHHS dataset. Additionally, it has reduced computational requirements, with only 0.27M parameters and 0.35G floating-point operations, and it claims to be faster than models like L-SeqSleepNet.
    
    \item[\textbf{[2]}] The Multi-Domain View Self-Supervised Learning Framework (MV-TTFC) introduces a new approach to classify sleep stages by leveraging self-supervised learning (SSL) on unlabeled EEG data. By incorporating multi-view representation technology, this model enhances information exchange across different views. It also introduces the multisynchrosqueezing transform, which improves the quality of the time-frequency view. Ultimately, it captures the latent features within EEG signals. It was evaluated on two datasets (SleepEDF-78 and SHHS), and MV-TTFC achieved state-of-the-art performance with accuracies of 78.64\% and 81.45\%, and macro F1-scores of 70.39\% and 70.47\%, respectively.
    
    \item[\textbf{[3]}] The proposed CNN-Transformer-ConvLSTM-CRF hybrid model presents a new integration method between local and global feature extraction to enhance the classification ability of sleep stages. The model can identify relationships among EEG features by applying a multi-scale convolutional neural network combined with a Transformer for encoding features of the EEG signal and a spatio-temporal encoder via ConvLSTM. Additionally, the adaptive feature calibration module improves the extracted features, and there is efficient learning of the transition relationships between the stages of sleep by the CRF module. Based on evaluations on three datasets, this hybrid model outperforms existing state-of-the-art methods, demonstrating its efficacy in sleep stage classification.
\end{enumerate}


